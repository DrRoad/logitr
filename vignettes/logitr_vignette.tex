\documentclass[article]{jss}
\usepackage[utf8]{inputenc}

\providecommand{\tightlist}{%
  \setlength{\itemsep}{0pt}\setlength{\parskip}{0pt}}

\author{
John Paul Helveston\\George Washington University
}
\title{Flexible Multinomial Logit Models with Preference Space and
Willingness-to-Pay Space Utility Specifications in R: The \pkg{logitr}
Package}

\Plainauthor{John Paul Helveston}
\Plaintitle{Flexible Multinomial Logit Models with Preference Space and
Willingness-to-Pay Space Utility Specifications in R: The logitr Package}
\Shorttitle{\pkg{logitr}: Preference and WTP Space Multinomial Logit Models}

\Abstract{
In many applications of discrete choice models, modelers are interested
in estimating consumer's marginal ``willingness-to-pay'' (WTP) for
different attributes. WTP can computed by dividing the estimated
parameters of a utility model in the preference space by the price
parameter or by estimating a utility model in the WTP space. For
homogeneous models, these two procedures generally produce the same
estimates of WTP, but the same is not true for heterogeneous models
where model parameters are assumed to follow a specific distribution.
The \pkg{logitr} package was written to support flexible estimation of
multinomial logit models with preference space and WTP space utility
specifications. The package supports homogeneous multinomial logit (MNL)
and heterogeneous mixed logit (MXL) models, including support for normal
and log-normal parameter distributions. Since MXL models and models with
WTP space utility specifications are non-convex, an option is included
to run a multi-start optimization loop with random starting points in
each iteration. The package also includes a simulation function to
estimate the expected shares of a set of alternatives based on an
estimated model.
}

\Keywords{multinomial logit, preference space, willingness-to-pay space, discrete choice, \proglang{R}}
\Plainkeywords{multinomial logit, preference space, willingness-to-pay space, discrete choice, R}

%% publication information
%% \Volume{50}
%% \Issue{9}
%% \Month{June}
%% \Year{2012}
%% \Submitdate{}
%% \Acceptdate{2012-06-04}

\Address{
    John Paul Helveston\\
  George Washington University\\
  Science \& Engineering Hall\hfill\break 800 22nd St
  NW\hfill\break Washington, DC 20052\\
  E-mail: \email{jph@gwu.edu}\\
  URL: \url{http://jhelvy.com}\\~\\
  }

% Pandoc header

\usepackage{amsmath} \usepackage{upgreek} \usepackage{longtable}
\usepackage{booktabs} \usepackage{float} \usepackage{array}

\begin{document}

\newcommand{\betaVec}{\boldsymbol\upbeta}
\newcommand{\omegaVec}{\boldsymbol\upomega}
\newcommand{\zetaVec}{\boldsymbol\upzeta}
\newcommand{\deltaVec}{\boldsymbol\updelta}
\newcommand{\gammaVec}{\boldsymbol\upgamma}
\newcommand{\epsilonVec}{\boldsymbol\upepsilon}
\newcommand{\xVec}{\mathrm{\mathbf{x}}}
\newcommand{\XVec}{\mathrm{\mathbf{X}}}

\begin{center}
\textcolor{red}{**WARNING: This document is not complete and may contain errors**}
\end{center}

\hypertarget{introduction}{%
\section{Introduction}\label{introduction}}

In many applications of discrete choice models, modelers are interested
in estimating consumer's marginal ``willingness-to-pay'' (WTP) for
different attributes. WTP can be estimated in two ways:

\begin{enumerate}
\def\labelenumi{\arabic{enumi}.}
\tightlist
\item
  Estimate a discrete choice model in the ``preference space'' where
  parameters have units of utility and then compute the WTP by dividing
  the parameters by the price parameter.
\item
  Estimate a discrete choice model in the ``WTP space'' where parameters
  have units of WTP.
\end{enumerate}

While the two procedures generally produce the same estimates of WTP for
homogenous models, the same is not true for heterogeneous models where
model parameters are assumed to follow a specific distribution, such as
normal or log-normal \citep{Train2005}. For example, in a preference
space specification, a normally distributed attribute parameter divided
by a log-normally distributed price parameter produces a strange WTP
distribution with large tails. In contrast, a WTP space specification
allows the modeler to directly assume WTP is normally distributed. The
\pkg{logitr} package was developed to enable modelers to choose between
these two utility spaces when estimating multinomial logit models.

\hypertarget{the-random-utility-model-in-two-spaces}{%
\section{The random utility model in two
spaces}\label{the-random-utility-model-in-two-spaces}}

The random utility model is a well-established framework in many fields
for estimating consumer preferences from observed consumer choices
\citep[\citet{Train2009}]{Louviere2000}. Random utility models assume
that consumers choose the alternative \(j\) a set of alternatives that
has the greatest utility \(u_{j}\). Utility is a random variable that is
modeled as \(u_{j} = v_{j} + \varepsilon_{j}\), where \(v_{j}\) is the
``observed utility'' (a function of the observed attributes such that
\(v_{j} = f(\mathrm{\mathbf{x}}_{j})\)) and \(\varepsilon_{j}\) is a
random variable representing the portion of utility unobservable to the
modeler.

Adopting the same notation as in Helveston et al.
\citeyearpar{Helveston2018}, consider the following utility model:

\input{./eqns/utility.Rmd}

where \(\boldsymbol\upbeta^{*}\) is the vector of coefficients for
non-price attributes \(\mathrm{\mathbf{x}}_{j}\), \(\alpha^{*}\) is the
coefficient for price \(p_{j}\), and the error term,
\(\varepsilon^{*}_{j}\), is an IID random variable with a Gumbel extreme
value distribution of mean zero and variance \(\sigma^2(\pi^2/6)\). This
model is not identified since there exists an infinite set of
combinations of values for \(\boldsymbol\upbeta^{*}\), \(\alpha^{*}\),
and \(\sigma\) that produce the same choice probabilities. In order to
specify an identifiable model, the modeler must normalize equation
(\ref{eq:utility}). One approach is to normalize the scale of the error
term by dividing equation (\ref{eq:utility}) by \(\sigma\), producing
the ``preference space'' utility specification:

\input{./eqns/utilityPreferenceScaled.Rmd}

The typical preference space parameterization of the multinomial logit
(MNL) model can then be written by rewriting equation
(\ref{eq:utilityPreferenceScaled}) with \(u_j = (u^*_j / \sigma)\),
\(\boldsymbol\upbeta= (\boldsymbol\upbeta^{*} / \sigma)\),
\(\alpha = (\alpha^{*} / \sigma)\), and
\(\varepsilon_{j} = (\varepsilon^{*}_{j} / \sigma)\):

\input{./eqns/utilityPreference.Rmd}

The vector \(\boldsymbol\upbeta\) represents the marginal utility for
changes in each non-price attribute, and \(\alpha\) represents the
marginal utility obtained from price reductions. In addition, the
coefficients \(\boldsymbol\upbeta\) and \(\alpha\) are measured in units
of \emph{utility}, which only has relative rather than absolute meaning.

The alternative normalization approach is to normalize equation
(\ref{eq:utility}) by \(\alpha^*\) instead of \(\sigma\), producing the
``willingness-to-pay (WTP) space'' utility specification:

\input{./eqns/utilityWtpScaled.Rmd}

Since the error term in equation is scaled by
\(\lambda^2 = \sigma^2/(\alpha^{*})^2\), we can rewrite equation
(\ref{eq:utilityWtpScaled}) by multiplying both sides by
\(\lambda= (\alpha^{*} / \sigma\)) and renaming
\(u_j = (\lambda u^*_j / \alpha^*)\),
\(\boldsymbol\upomega= (\boldsymbol\upbeta^{*} / \alpha^{*}\)), and
\(\varepsilon_j = (\lambda \varepsilon^*_j / \alpha^*)\):

\input{./eqns/utilityWtp.Rmd}

Here \(\boldsymbol\upomega\) represents the marginal WTP for changes in
each non-price attribute, and \(\lambda\) represents the scale of the
deterministic portion of utility relative to the standardized scale of
the random error term.

The utility models in equations \ref{eq:utilityPreference} and
\ref{eq:utilityWtp} represent the preference space and WTP space utility
specifications, respectively. In equation \ref{eq:utilityPreference},
WTP is estimated as \(\hat{\boldsymbol\upbeta} / \hat{\alpha}\); in
equation \ref{eq:utilityWtp}, WTP is simply
\(\hat{\boldsymbol\upomega}\).

\newpage

\hypertarget{using-the-logitr-package}{%
\section{Using the logitr package}\label{using-the-logitr-package}}

\hypertarget{installation}{%
\subsection{Installation}\label{installation}}

This package has not been uploaded to CRAN, but it can be directly
installed from Github using the \pkg{devtools} library. The package also
depends on the \pkg{nloptr} library.

First, make sure you have the \pkg{devtools} and \pkg{nloptr} libraries
installed:

\begin{quote}
\texttt{install.packages("devtools")}~\\
\texttt{install.packages("nloptr")}
\end{quote}

Then load the \pkg{devtools} library and install the \pkg{logitr}
package:

\begin{quote}
\texttt{library("devtools")}~\\
\texttt{install\_github("jhelvy/logitr")}
\end{quote}

\hypertarget{data-format}{%
\subsection{Data format}\label{data-format}}

The data must be arranged the following way:

\begin{enumerate}
\def\labelenumi{\arabic{enumi}.}
\tightlist
\item
  The data must be a \texttt{data.frame} object.
\item
  Each row is an alternative from a choice observation. Each choice
  observation does not have to have the same number of alternatives.
\item
  Each column is a variable.
\item
  One column must identify \texttt{obsID} (the ``observation ID''): a
  sequence of numbers that identifies each unique choice occasion. For
  example, if the first three choice occasions had 2 alternatives each,
  then the first 9 rows of the \texttt{obsID} variable would be
  \texttt{1,1,2,2,3,3}.
\item
  One column must identify \texttt{choice}: a dummy variable that
  identifies which alternative was chosen (\texttt{1}=chosen,
  \texttt{0}=not chosen).
\item
  For WTP space models, once column must identify \texttt{price}: a
  continous variable of the price values.
\end{enumerate}

An example of of the \texttt{Yogurt} data set from the \pkg{mlogit}
package illustrates this format:

\begin{CodeChunk}

\begin{CodeInput}
R> library("logitr")
R> data(yogurt)
R> head(yogurt, 12)
\end{CodeInput}

\begin{CodeOutput}
   id obsID choice price feat   brand dannon hiland weight yoplait
1   1     1      0   8.1    0  dannon      1      0      0       0
2   1     1      0   6.1    0  hiland      0      1      0       0
3   1     1      1   7.9    0  weight      0      0      1       0
4   1     1      0  10.8    0 yoplait      0      0      0       1
5   1     2      1   9.8    0  dannon      1      0      0       0
6   1     2      0   6.4    0  hiland      0      1      0       0
7   1     2      0   7.5    0  weight      0      0      1       0
8   1     2      0  10.8    0 yoplait      0      0      0       1
9   1     3      1   9.8    0  dannon      1      0      0       0
10  1     3      0   6.1    0  hiland      0      1      0       0
11  1     3      0   8.6    0  weight      0      0      1       0
12  1     3      0  10.8    0 yoplait      0      0      0       1
\end{CodeOutput}
\end{CodeChunk}

\hypertarget{the-logitr-function}{%
\subsection{The logitr() function}\label{the-logitr-function}}

The main model estimation function is the \texttt{logitr()} function:

\begin{CodeChunk}

\begin{CodeInput}
R> model = logitr(data, choiceName, obsIDName, parNames, priceName=NULL,
R>             randPars=NULL, randPrice=NULL, modelSpace="pref",
R>             options=list(...))
\end{CodeInput}
\end{CodeChunk}

The function returns a list of values, so assign the model output to a
variable (e.g. \texttt{model}) to store the output values.

\hypertarget{arguments}{%
\subsubsection{Arguments}\label{arguments}}

\begin{longtable}[]{@{}lll@{}}
\toprule
\begin{minipage}[b]{0.14\columnwidth}\raggedright
Argument\strut
\end{minipage} & \begin{minipage}[b]{0.66\columnwidth}\raggedright
Description\strut
\end{minipage} & \begin{minipage}[b]{0.11\columnwidth}\raggedright
Default\strut
\end{minipage}\tabularnewline
\midrule
\endhead
\begin{minipage}[t]{0.14\columnwidth}\raggedright
\texttt{data}\strut
\end{minipage} & \begin{minipage}[t]{0.66\columnwidth}\raggedright
The choice data, formatted as a data.frame object.\strut
\end{minipage} & \begin{minipage}[t]{0.11\columnwidth}\raggedright
--\strut
\end{minipage}\tabularnewline
\begin{minipage}[t]{0.14\columnwidth}\raggedright
\texttt{choiceName}\strut
\end{minipage} & \begin{minipage}[t]{0.66\columnwidth}\raggedright
The name of the column that identifies the \texttt{choice}
variable.\strut
\end{minipage} & \begin{minipage}[t]{0.11\columnwidth}\raggedright
--\strut
\end{minipage}\tabularnewline
\begin{minipage}[t]{0.14\columnwidth}\raggedright
\texttt{obsIDName}\strut
\end{minipage} & \begin{minipage}[t]{0.66\columnwidth}\raggedright
The name of the column that identifies the \texttt{obsID}
variable.\strut
\end{minipage} & \begin{minipage}[t]{0.11\columnwidth}\raggedright
--\strut
\end{minipage}\tabularnewline
\begin{minipage}[t]{0.14\columnwidth}\raggedright
\texttt{parNames}\strut
\end{minipage} & \begin{minipage}[t]{0.66\columnwidth}\raggedright
The names of the parameters to be estimated in the model. Must be the
same as the column names in the \texttt{data} argument. For WTP space
models, do not include \texttt{price} in \texttt{parNames}.\strut
\end{minipage} & \begin{minipage}[t]{0.11\columnwidth}\raggedright
--\strut
\end{minipage}\tabularnewline
\begin{minipage}[t]{0.14\columnwidth}\raggedright
\texttt{priceName}\strut
\end{minipage} & \begin{minipage}[t]{0.66\columnwidth}\raggedright
The name of the column that identifies the price variable. Only required
for WTP space models.\strut
\end{minipage} & \begin{minipage}[t]{0.11\columnwidth}\raggedright
\texttt{NULL}\strut
\end{minipage}\tabularnewline
\begin{minipage}[t]{0.14\columnwidth}\raggedright
\texttt{randPars}\strut
\end{minipage} & \begin{minipage}[t]{0.66\columnwidth}\raggedright
A named vector whose names are the random parameters and values the
destribution: \texttt{\textquotesingle{}n\textquotesingle{}} for normal
or \texttt{\textquotesingle{}ln\textquotesingle{}} for log-normal.\strut
\end{minipage} & \begin{minipage}[t]{0.11\columnwidth}\raggedright
\texttt{NULL}\strut
\end{minipage}\tabularnewline
\begin{minipage}[t]{0.14\columnwidth}\raggedright
\texttt{randPrice}\strut
\end{minipage} & \begin{minipage}[t]{0.66\columnwidth}\raggedright
The random distribution for the price parameter:
\texttt{\textquotesingle{}n\textquotesingle{}} for normal or
\texttt{\textquotesingle{}ln\textquotesingle{}} for log-normal. Only
used for WTP space MXL models.\strut
\end{minipage} & \begin{minipage}[t]{0.11\columnwidth}\raggedright
\texttt{NULL}\strut
\end{minipage}\tabularnewline
\begin{minipage}[t]{0.14\columnwidth}\raggedright
\texttt{modelSpace}\strut
\end{minipage} & \begin{minipage}[t]{0.66\columnwidth}\raggedright
Set to \texttt{\textquotesingle{}wtp\textquotesingle{}} for WTP space
models.\strut
\end{minipage} & \begin{minipage}[t]{0.11\columnwidth}\raggedright
\texttt{\textquotesingle{}pref\textquotesingle{}}\strut
\end{minipage}\tabularnewline
\begin{minipage}[t]{0.14\columnwidth}\raggedright
\texttt{options}\strut
\end{minipage} & \begin{minipage}[t]{0.66\columnwidth}\raggedright
A list of options.\strut
\end{minipage} & \begin{minipage}[t]{0.11\columnwidth}\raggedright
--\strut
\end{minipage}\tabularnewline
\bottomrule
\end{longtable}

\hypertarget{options}{%
\subsubsection{Options}\label{options}}

\begin{longtable}[]{@{}lll@{}}
\toprule
\begin{minipage}[b]{0.18\columnwidth}\raggedright
Argument\strut
\end{minipage} & \begin{minipage}[b]{0.61\columnwidth}\raggedright
Description\strut
\end{minipage} & \begin{minipage}[b]{0.12\columnwidth}\raggedright
Default\strut
\end{minipage}\tabularnewline
\midrule
\endhead
\begin{minipage}[t]{0.18\columnwidth}\raggedright
\texttt{numMultiStarts}\strut
\end{minipage} & \begin{minipage}[t]{0.61\columnwidth}\raggedright
Number of times to run the optimization loop, each time starting from a
different random starting point for each parameter between
\texttt{startParBounds}. Recommended for non-convex models, such as WTP
space models and MXL models.\strut
\end{minipage} & \begin{minipage}[t]{0.12\columnwidth}\raggedright
\texttt{1}\strut
\end{minipage}\tabularnewline
\begin{minipage}[t]{0.18\columnwidth}\raggedright
\texttt{keepAllRuns}\strut
\end{minipage} & \begin{minipage}[t]{0.61\columnwidth}\raggedright
Set to \texttt{TRUE} to keep all the model information for each
multistart run. If \texttt{TRUE}, the \texttt{logitr()} function will
return a list with two values: \texttt{models} (a list of each model),
and \texttt{bestModel} (the model with the largest log-likelihood
value).\strut
\end{minipage} & \begin{minipage}[t]{0.12\columnwidth}\raggedright
\texttt{FALSE}\strut
\end{minipage}\tabularnewline
\begin{minipage}[t]{0.18\columnwidth}\raggedright
\texttt{startParBounds}\strut
\end{minipage} & \begin{minipage}[t]{0.61\columnwidth}\raggedright
Set the \texttt{lower} and \texttt{upper} bounds for the starting
parameters for each optimization run, which are generated by
\texttt{runif(n,\ lower,\ upper)}.\strut
\end{minipage} & \begin{minipage}[t]{0.12\columnwidth}\raggedright
\texttt{c(-1,1)}\strut
\end{minipage}\tabularnewline
\begin{minipage}[t]{0.18\columnwidth}\raggedright
\texttt{startVals}\strut
\end{minipage} & \begin{minipage}[t]{0.61\columnwidth}\raggedright
A vector of values to be used as starting values for the optimization.
Only used for the first run if
\texttt{numMultiStarts\ \textgreater{}\ 1}.\strut
\end{minipage} & \begin{minipage}[t]{0.12\columnwidth}\raggedright
\texttt{NULL}\strut
\end{minipage}\tabularnewline
\begin{minipage}[t]{0.18\columnwidth}\raggedright
\texttt{useAnalyticGrad}\strut
\end{minipage} & \begin{minipage}[t]{0.61\columnwidth}\raggedright
Set to \texttt{FALSE} to use numerically approximated gradients instead
of analytic gradients during estimation (which is slower).\strut
\end{minipage} & \begin{minipage}[t]{0.12\columnwidth}\raggedright
\texttt{TRUE}\strut
\end{minipage}\tabularnewline
\begin{minipage}[t]{0.18\columnwidth}\raggedright
\texttt{scaleInputs}\strut
\end{minipage} & \begin{minipage}[t]{0.61\columnwidth}\raggedright
By default each variable in \texttt{data} is scaled to be between 0 and
1 before running the optimization routine because it usually helps with
stability, especially if some of the variables have very large or very
small values (e.g. \texttt{\textgreater{}\ 10\^{}3} or
\texttt{\textless{}\ 10\^{}-3}). Set to \texttt{FALSE} to turn this
feature off.\strut
\end{minipage} & \begin{minipage}[t]{0.12\columnwidth}\raggedright
\texttt{TRUE}\strut
\end{minipage}\tabularnewline
\begin{minipage}[t]{0.18\columnwidth}\raggedright
\texttt{standardDraws}\strut
\end{minipage} & \begin{minipage}[t]{0.61\columnwidth}\raggedright
By default, a new set of standard normal draws are generated during each
call to \texttt{logitr} (the same draws are used during each multistart
too). The user can override those draws by providing a matrix of
standard normal draws if desired.\strut
\end{minipage} & \begin{minipage}[t]{0.12\columnwidth}\raggedright
\texttt{NULL}\strut
\end{minipage}\tabularnewline
\begin{minipage}[t]{0.18\columnwidth}\raggedright
\texttt{numDraws}\strut
\end{minipage} & \begin{minipage}[t]{0.61\columnwidth}\raggedright
The number of draws to use for MXL models for the maximum simulated
likelihood.\strut
\end{minipage} & \begin{minipage}[t]{0.12\columnwidth}\raggedright
\texttt{200}\strut
\end{minipage}\tabularnewline
\begin{minipage}[t]{0.18\columnwidth}\raggedright
\texttt{drawType}\strut
\end{minipage} & \begin{minipage}[t]{0.61\columnwidth}\raggedright
The type of draw to use for MXL models for the maximum simulated
likelihood. Set to \texttt{\textquotesingle{}normal\textquotesingle{}}
to use random normal draws or
\texttt{\textquotesingle{}halton\textquotesingle{}} for Halton
draws.\strut
\end{minipage} & \begin{minipage}[t]{0.12\columnwidth}\raggedright
\texttt{\textquotesingle{}halton\textquotesingle{}}\strut
\end{minipage}\tabularnewline
\begin{minipage}[t]{0.18\columnwidth}\raggedright
\texttt{printLevel}\strut
\end{minipage} & \begin{minipage}[t]{0.61\columnwidth}\raggedright
The print level of the \texttt{nloptr} optimization loop. Type
\texttt{nloptr.print.options()} for more details.\strut
\end{minipage} & \begin{minipage}[t]{0.12\columnwidth}\raggedright
\texttt{0}\strut
\end{minipage}\tabularnewline
\begin{minipage}[t]{0.18\columnwidth}\raggedright
\texttt{xtol\_rel}\strut
\end{minipage} & \begin{minipage}[t]{0.61\columnwidth}\raggedright
The relative \texttt{x} tolerance for the \texttt{nloptr} optimization
loop. Type \texttt{nloptr.print.options()} for more details.\strut
\end{minipage} & \begin{minipage}[t]{0.12\columnwidth}\raggedright
\texttt{1.0e-8}\strut
\end{minipage}\tabularnewline
\begin{minipage}[t]{0.18\columnwidth}\raggedright
\texttt{xtol\_abs}\strut
\end{minipage} & \begin{minipage}[t]{0.61\columnwidth}\raggedright
The absolute \texttt{x} tolerance for the \texttt{nloptr} optimization
loop. Type \texttt{nloptr.print.options()} for more details.\strut
\end{minipage} & \begin{minipage}[t]{0.12\columnwidth}\raggedright
\texttt{1.0e-8}\strut
\end{minipage}\tabularnewline
\begin{minipage}[t]{0.18\columnwidth}\raggedright
\texttt{ftol\_rel}\strut
\end{minipage} & \begin{minipage}[t]{0.61\columnwidth}\raggedright
The relative \texttt{f} tolerance for the \texttt{nloptr} optimization
loop. Type \texttt{nloptr.print.options()} for more details.\strut
\end{minipage} & \begin{minipage}[t]{0.12\columnwidth}\raggedright
\texttt{1.0e-8}\strut
\end{minipage}\tabularnewline
\begin{minipage}[t]{0.18\columnwidth}\raggedright
\texttt{ftol\_abs}\strut
\end{minipage} & \begin{minipage}[t]{0.61\columnwidth}\raggedright
The absolute \texttt{f} tolerance for the \texttt{nloptr} optimization
loop. Type \texttt{nloptr.print.options()} for more details.\strut
\end{minipage} & \begin{minipage}[t]{0.12\columnwidth}\raggedright
\texttt{1.0e-8}\strut
\end{minipage}\tabularnewline
\begin{minipage}[t]{0.18\columnwidth}\raggedright
\texttt{maxeval}\strut
\end{minipage} & \begin{minipage}[t]{0.61\columnwidth}\raggedright
The maximum number of function evaluations for the \texttt{nloptr}
optimization loop. Type \texttt{nloptr.print.options()} for more
details.\strut
\end{minipage} & \begin{minipage}[t]{0.12\columnwidth}\raggedright
\texttt{1000}\strut
\end{minipage}\tabularnewline
\bottomrule
\end{longtable}

\hypertarget{values}{%
\subsubsection{Values}\label{values}}

\begin{longtable}[]{@{}ll@{}}
\toprule
\begin{minipage}[b]{0.21\columnwidth}\raggedright
Value\strut
\end{minipage} & \begin{minipage}[b]{0.73\columnwidth}\raggedright
Description\strut
\end{minipage}\tabularnewline
\midrule
\endhead
\begin{minipage}[t]{0.21\columnwidth}\raggedright
\texttt{coef}\strut
\end{minipage} & \begin{minipage}[t]{0.73\columnwidth}\raggedright
The model coefficients at convergence.\strut
\end{minipage}\tabularnewline
\begin{minipage}[t]{0.21\columnwidth}\raggedright
\texttt{standErrs}\strut
\end{minipage} & \begin{minipage}[t]{0.73\columnwidth}\raggedright
The standard errors of the model coefficients at convergence.\strut
\end{minipage}\tabularnewline
\begin{minipage}[t]{0.21\columnwidth}\raggedright
\texttt{logLik}\strut
\end{minipage} & \begin{minipage}[t]{0.73\columnwidth}\raggedright
The log-likelihood value at convergence.\strut
\end{minipage}\tabularnewline
\begin{minipage}[t]{0.21\columnwidth}\raggedright
\texttt{nullLogLik}\strut
\end{minipage} & \begin{minipage}[t]{0.73\columnwidth}\raggedright
The null log-likelihood value (if all coefficients are 0).\strut
\end{minipage}\tabularnewline
\begin{minipage}[t]{0.21\columnwidth}\raggedright
\texttt{gradient}\strut
\end{minipage} & \begin{minipage}[t]{0.73\columnwidth}\raggedright
The gradient of the log-likelihood at convergence.\strut
\end{minipage}\tabularnewline
\begin{minipage}[t]{0.21\columnwidth}\raggedright
\texttt{hessian}\strut
\end{minipage} & \begin{minipage}[t]{0.73\columnwidth}\raggedright
The hessian of the log-likelihood at convergence.\strut
\end{minipage}\tabularnewline
\begin{minipage}[t]{0.21\columnwidth}\raggedright
\texttt{numObs}\strut
\end{minipage} & \begin{minipage}[t]{0.73\columnwidth}\raggedright
The number of observations.\strut
\end{minipage}\tabularnewline
\begin{minipage}[t]{0.21\columnwidth}\raggedright
\texttt{numParams}\strut
\end{minipage} & \begin{minipage}[t]{0.73\columnwidth}\raggedright
The number of model parameters.\strut
\end{minipage}\tabularnewline
\begin{minipage}[t]{0.21\columnwidth}\raggedright
\texttt{startPars}\strut
\end{minipage} & \begin{minipage}[t]{0.73\columnwidth}\raggedright
The starting values used.\strut
\end{minipage}\tabularnewline
\begin{minipage}[t]{0.21\columnwidth}\raggedright
\texttt{multistartNumber}\strut
\end{minipage} & \begin{minipage}[t]{0.73\columnwidth}\raggedright
The multistart run number for this model.\strut
\end{minipage}\tabularnewline
\begin{minipage}[t]{0.21\columnwidth}\raggedright
\texttt{time}\strut
\end{minipage} & \begin{minipage}[t]{0.73\columnwidth}\raggedright
The user, system, and elapsed time to run the optimization.\strut
\end{minipage}\tabularnewline
\begin{minipage}[t]{0.21\columnwidth}\raggedright
\texttt{iterations}\strut
\end{minipage} & \begin{minipage}[t]{0.73\columnwidth}\raggedright
The number of iterations until convergence.\strut
\end{minipage}\tabularnewline
\begin{minipage}[t]{0.21\columnwidth}\raggedright
\texttt{message}\strut
\end{minipage} & \begin{minipage}[t]{0.73\columnwidth}\raggedright
A more informative message with the status of the optimization
result.\strut
\end{minipage}\tabularnewline
\begin{minipage}[t]{0.21\columnwidth}\raggedright
\texttt{status}\strut
\end{minipage} & \begin{minipage}[t]{0.73\columnwidth}\raggedright
An integer value with the status of the optimization (positive values
are successes). Type \texttt{logitr.statusCodes()} for a detailed
description.\strut
\end{minipage}\tabularnewline
\begin{minipage}[t]{0.21\columnwidth}\raggedright
\texttt{modelSpace}\strut
\end{minipage} & \begin{minipage}[t]{0.73\columnwidth}\raggedright
The model space (\texttt{\textquotesingle{}pref\textquotesingle{}} or
\texttt{\textquotesingle{}wtp\textquotesingle{}}).\strut
\end{minipage}\tabularnewline
\begin{minipage}[t]{0.21\columnwidth}\raggedright
\texttt{standardDraws}\strut
\end{minipage} & \begin{minipage}[t]{0.73\columnwidth}\raggedright
The draws used during maximum simulated likelihood (for MXL
models).\strut
\end{minipage}\tabularnewline
\begin{minipage}[t]{0.21\columnwidth}\raggedright
\texttt{randParSummary}\strut
\end{minipage} & \begin{minipage}[t]{0.73\columnwidth}\raggedright
A summary of any random parameters (for MXL models).\strut
\end{minipage}\tabularnewline
\begin{minipage}[t]{0.21\columnwidth}\raggedright
\texttt{parSetup}\strut
\end{minipage} & \begin{minipage}[t]{0.73\columnwidth}\raggedright
A summary of the distributional assumptions on each model parameter
(\texttt{"f"}=``fixed'', \texttt{"n"}=``normal distribution'',
\texttt{"ln"}=``log-normal distribution'').\strut
\end{minipage}\tabularnewline
\begin{minipage}[t]{0.21\columnwidth}\raggedright
\texttt{options}\strut
\end{minipage} & \begin{minipage}[t]{0.73\columnwidth}\raggedright
A list of all the model options.\strut
\end{minipage}\tabularnewline
\bottomrule
\end{longtable}

\hypertarget{details-about-parnames-argument}{%
\subsection{Details about ``parNames''
argument:}\label{details-about-parnames-argument}}

A structural assumption in the \texttt{logitr} package is that the
deterministic part of the utility specification is linear in parameters:
\(v_{j} = \boldsymbol\upbeta' \mathrm{\mathbf{x}}_{j}\) for preference
space models, and
\(v_{j} = \boldsymbol\upomega' \mathrm{\mathbf{x}}_{j}\) for WTP space
models. Accordingly, each parameter in the \texttt{parNames} argument
must correspond to a variable in the data that is an additive part of
\(v_{j}\). For WTP space models, the \texttt{parNames} should only
include the WTP parameters, and the \texttt{price} parameter is denoted
by the separate argument \texttt{priceName}. Here are several examples:

\begin{longtable}[]{@{}llll@{}}
\toprule
\begin{minipage}[b]{0.13\columnwidth}\raggedright
Space\strut
\end{minipage} & \begin{minipage}[b]{0.39\columnwidth}\raggedright
Model\strut
\end{minipage} & \begin{minipage}[b]{0.24\columnwidth}\raggedright
\texttt{parNames}\strut
\end{minipage} & \begin{minipage}[b]{0.13\columnwidth}\raggedright
\texttt{priceName}\strut
\end{minipage}\tabularnewline
\midrule
\endhead
\begin{minipage}[t]{0.13\columnwidth}\raggedright
Preference\strut
\end{minipage} & \begin{minipage}[t]{0.39\columnwidth}\raggedright
\(u_{j} = \beta_1 price_j + \beta_2 size_j + \varepsilon_{j}\)\strut
\end{minipage} & \begin{minipage}[t]{0.24\columnwidth}\raggedright
\texttt{c(\textquotesingle{}price\textquotesingle{},\ \textquotesingle{}size\textquotesingle{})}\strut
\end{minipage} & \begin{minipage}[t]{0.13\columnwidth}\raggedright
\texttt{NULL}\strut
\end{minipage}\tabularnewline
\begin{minipage}[t]{0.13\columnwidth}\raggedright
WTP\strut
\end{minipage} & \begin{minipage}[t]{0.39\columnwidth}\raggedright
\(u_{j} = \lambda_j (\beta_1 size_j - price_j) + \varepsilon_{j}\)\strut
\end{minipage} & \begin{minipage}[t]{0.24\columnwidth}\raggedright
\texttt{c(\textquotesingle{}size\textquotesingle{})}\strut
\end{minipage} & \begin{minipage}[t]{0.13\columnwidth}\raggedright
\texttt{\textquotesingle{}price\textquotesingle{}}\strut
\end{minipage}\tabularnewline
\bottomrule
\end{longtable}

\hypertarget{using-summary-with-logitr}{%
\subsection{Using summary() with
logitr}\label{using-summary-with-logitr}}

The \texttt{logitr} package includes a summary function that has several
variations:

\begin{itemize}
\tightlist
\item
  For a single model run, it prints some summary information, including
  the model space, log-likelihood value at the solution, and a summary
  table of the model coefficients.
\item
  For MXL models, the function also prints a summary of the random
  parameters.
\item
  If the \texttt{keepAllRuns} option is set to \texttt{TRUE}, the
  function will print a summary of all the multistart runs followed by a
  summary of the best model (as determined by the largest log-likelihood
  value).
\end{itemize}

To understand the status code of any model, type
\texttt{logitr.statusCodes()}, which prints a description of each status
code from the \texttt{nloptr} optimization routine.

\hypertarget{computing-and-comparing-wtp}{%
\subsection{Computing and Comparing
WTP}\label{computing-and-comparing-wtp}}

For models in the preference space, you can get a summary table of the
implied WTP by using the \texttt{wtp()} function:

\begin{quote}
\texttt{wtp(model,\ priceName)}
\end{quote}

To compare the WTP between two equivalent models in the preference space
and WTP spaces, use the \texttt{wtpCompare()} function:

\begin{quote}
\texttt{wtpCompare(prefSpaceModel,\ wtpSpaceModel,\ priceName)}
\end{quote}

\hypertarget{simulation}{%
\subsection{Simulation}\label{simulation}}

After estimating a model, often times modelers want to use the results
to simulate the expected shares of a particular set of alternatives.
This can be done using the function \texttt{simulateShares()}. The
simulation reports the expected share as well as a confidence interval
for each alternative:

\begin{quote}
\texttt{shares\ =\ simulateShares(model,\ alts,\ priceName=NULL,\ alpha=0.025)}
\end{quote}

\hypertarget{arguments-1}{%
\subsubsection{Arguments}\label{arguments-1}}

\begin{longtable}[]{@{}lll@{}}
\toprule
\begin{minipage}[b]{0.14\columnwidth}\raggedright
Argument\strut
\end{minipage} & \begin{minipage}[b]{0.66\columnwidth}\raggedright
Description\strut
\end{minipage} & \begin{minipage}[b]{0.11\columnwidth}\raggedright
Default\strut
\end{minipage}\tabularnewline
\midrule
\endhead
\begin{minipage}[t]{0.14\columnwidth}\raggedright
\texttt{model}\strut
\end{minipage} & \begin{minipage}[t]{0.66\columnwidth}\raggedright
A MNL or MXL model estimated using the \texttt{logitr} package.\strut
\end{minipage} & \begin{minipage}[t]{0.11\columnwidth}\raggedright
--\strut
\end{minipage}\tabularnewline
\begin{minipage}[t]{0.14\columnwidth}\raggedright
\texttt{alts}\strut
\end{minipage} & \begin{minipage}[t]{0.66\columnwidth}\raggedright
A data frame of the alternatives. Each row should be an alternative, and
each column an attribute for which there is a corresponding coefficient
in the estimated model.\strut
\end{minipage} & \begin{minipage}[t]{0.11\columnwidth}\raggedright
--\strut
\end{minipage}\tabularnewline
\begin{minipage}[t]{0.14\columnwidth}\raggedright
\texttt{priceName}\strut
\end{minipage} & \begin{minipage}[t]{0.66\columnwidth}\raggedright
The name of the column in \texttt{alts} that identifies price (only
required for WTP space models).\strut
\end{minipage} & \begin{minipage}[t]{0.11\columnwidth}\raggedright
\texttt{NULL}\strut
\end{minipage}\tabularnewline
\begin{minipage}[t]{0.14\columnwidth}\raggedright
\texttt{alpha}\strut
\end{minipage} & \begin{minipage}[t]{0.66\columnwidth}\raggedright
The significance level for the confidence interval (e.g. \texttt{0.025}
results in a 95\% CI).\strut
\end{minipage} & \begin{minipage}[t]{0.11\columnwidth}\raggedright
\texttt{0.025}\strut
\end{minipage}\tabularnewline
\bottomrule
\end{longtable}

\hypertarget{citation-information}{%
\subsection{Citation Information}\label{citation-information}}

If you use this package for an analysis that is published, I would
greatly appreciate it if you included a citation. You can get the
citation information by using the \texttt{citation()} function:

\begin{CodeChunk}

\begin{CodeInput}
R> citation("logitr")
\end{CodeInput}

\begin{CodeOutput}

To cite package 'logitr' in publications use:

  John Helveston (2018). logitr: Multinomial and mixed logit
  estimation in preference and willingness to pay space utility
  specifications. R package version 1.2.

A BibTeX entry for LaTeX users is

  @Manual{,
    title = {logitr: Multinomial and mixed logit estimation in preference and willingness to pay space utility specifications},
    author = {John Helveston},
    year = {2018},
    note = {R package version 1.2},
  }
\end{CodeOutput}
\end{CodeChunk}

\newpage

\hypertarget{examples}{%
\section{Examples}\label{examples}}

All examples use the \texttt{Yogurt} data set from Jain et al.
\citeyearpar{Jain1994}, reformatted for use in the \texttt{logitr}
package. The data set contains 2,412 choice observations from a series
of yogurt purchases by a panel of 100 households in Springfield,
Missouri, over a roughly two-year period. The data were collected by
optical scanners and contain information about the price, brand, and a
``feature'' variable, which identifies whether a newspaper advertisement
was shown to the customer. There are four brands of yogurt: Yoplait,
Dannon, Weight Watchers, and Hiland, with market shares of 34\%, 40\%,
23\% and 3\%, respectively.

In the utility models described below, the data variables are
represented as follows:

\input{./tables/examplePars.Rmd}

\newpage

\hypertarget{mnl-model-in-the-preference-space}{%
\subsection{MNL model in the preference
space}\label{mnl-model-in-the-preference-space}}

Estimate the following homogeneous multinomial logit model in the
preference space:

\input{./eqns/mnlPrefExample.Rmd}

where the parameters \(\alpha\), \(\beta_1\), \(\beta_2\), \(\beta_3\),
and \(\beta_4\) have units of utility.

\textbf{Estimate the model}:

\begin{CodeChunk}

\begin{CodeInput}
R> library("logitr")
R> data(yogurt)
R> 
R> mnl.pref = logitr(
R>   data       = yogurt,
R>   choiceName = "choice",
R>   obsIDName  = "obsID",
R>   parNames   = c("price", "feat", "dannon", "hiland", "yoplait"))
\end{CodeInput}
\end{CodeChunk}

\textbf{Print a summary of the results}:

\begin{CodeChunk}

\begin{CodeInput}
R> summary(mnl.pref)
\end{CodeInput}

\begin{CodeOutput}
=================================================
MODEL SUMMARY: 
                         
Model Space:   Preference
Model Run:         1 of 1
Iterations:            26
Elapsed Time: 0h:0m:0.19s

Model Coefficients: 
         Estimate StdError    tStat pVal signif
price   -0.366584 0.024366 -15.0449    0    ***
feat     0.491432 0.120063   4.0931    0    ***
dannon   0.641186 0.054498  11.7652    0    ***
hiland  -3.074415 0.145384 -21.1468    0    ***
yoplait  1.375757 0.088982  15.4611    0    ***
---
Signif. codes:  0 '***' 0.001 '**' 0.01 '*' 0.05 '.' 0.1 ' ' 1

Model Fit Values: 
                                     
Log.Likelihood.         -2656.8878779
Null.Log.Likelihood.    -3343.7419990
AIC.                     5323.7758000
BIC.                     5352.7168000
McFadden.R2.                0.2054148
Adj..McFadden.R2            0.2039195
Number.of.Observations.  2412.0000000
\end{CodeOutput}
\end{CodeChunk}

\textbf{Get the estimated model coefficients}:

\begin{CodeChunk}

\begin{CodeInput}
R> coef(mnl.pref)
\end{CodeInput}

\begin{CodeOutput}
     price       feat     dannon     hiland    yoplait 
-0.3665844  0.4914317  0.6411857 -3.0744152  1.3757571 
\end{CodeOutput}
\end{CodeChunk}

\textbf{Get the WTP implied from the preference space model}:

\begin{CodeChunk}

\begin{CodeInput}
R> mnl.pref.wtp = wtp(mnl.pref, priceName="price")
R> mnl.pref.wtp
\end{CodeInput}

\begin{CodeOutput}
         Estimate StdError    tStat  pVal signif
lambda   0.366584 0.024307  15.0815 0e+00    ***
feat     1.340569 0.357830   3.7464 2e-04    ***
dannon   1.749081 0.198467   8.8130 0e+00    ***
hiland  -8.386651 0.508709 -16.4862 0e+00    ***
yoplait  3.752907 0.468238   8.0150 0e+00    ***
\end{CodeOutput}
\end{CodeChunk}

\newpage

\hypertarget{mnl-model-in-the-wtp-space}{%
\subsection{MNL model in the WTP
space}\label{mnl-model-in-the-wtp-space}}

Estimate the following homogeneous multinomial logit model in the WTP
space:

\input{./eqns/mnlWtpExample.Rmd}

where the parameters \(\omega_1\), \(\omega_2\), \(\omega_3\), and
\(\omega_4\) have units of dollars and \(\lambda\) is the scale
parameter.

\textbf{Estimate the model}:

\begin{CodeChunk}

\begin{CodeInput}
R> library("logitr")
R> data(yogurt)
R> 
R> # Use WTP from preference space model as starting values for first run:
\end{CodeInput}
\end{CodeChunk}

\texttt{startingValues\ =\ mnl.pref.wtp\$Estimate}

\begin{CodeChunk}

\begin{CodeInput}
R> mnl.wtp = logitr(
R>   data       = yogurt,
R>   choiceName = "choice",
R>   obsIDName  = "obsID",
R>   parNames   = c("feat", "dannon", "hiland", "yoplait"),
R>   priceName  = "price",
R>   modelSpace = "wtp",
R>   options = list(
R>     # Since WTP space models are non-convex, run a multistart:
R>     numMultiStarts = 10,
R>     # If you want to view the results from each multistart run,
R>     # set keepAllRuns=TRUE:
R>     keepAllRuns = TRUE,
R>     startVals   = startingValues,
R>     # Because the computed WTP from the preference space model has values
R>     # as large as 8, I increase the boundaries of the random starting values:
R>     startParBounds = c(-5,5)))
\end{CodeInput}
\end{CodeChunk}

\textbf{Print a summary of the results}:

\begin{CodeChunk}

\begin{CodeInput}
R> summary(mnl.wtp)
\end{CodeInput}

\begin{CodeOutput}
=================================================
SUMMARY OF ALL MULTISTART RUNS:

   run    logLik iterations status
1    1 -2656.888          6      3
2    2 -2841.899         69     -1
3    3 -2656.888         41      3
4    4 -2890.611         60     -1
5    5 -2819.873         71      3
6    6 -2885.528         58     -1
7    7 -2656.888         50      3
8    8 -2871.870         47     -1
9    9 -2656.888         43      3
10  10 -2656.888         41      3
---
To view meaning of status codes, use logitr.statusCodes() 

Summary of BEST model below (run with largest log-likelihood value)
=================================================
MODEL SUMMARY: 
                                
Model Space:  Willingness-to-Pay
Model Run:              10 of 10
Iterations:                   41
Elapsed Time:        0h:0m:0.23s

Model Coefficients: 
         Estimate StdError    tStat  pVal signif
lambda   0.366584 0.024366  15.0449 0e+00    ***
feat     1.340573 0.355865   3.7671 2e-04    ***
dannon   1.749077 0.179897   9.7227 0e+00    ***
hiland  -8.386650 0.502472 -16.6908 0e+00    ***
yoplait  3.752902 0.168121  22.3226 0e+00    ***
---
Signif. codes:  0 '***' 0.001 '**' 0.01 '*' 0.05 '.' 0.1 ' ' 1

Model Fit Values: 
                                     
Log.Likelihood.         -2656.8878779
Null.Log.Likelihood.    -3343.7419990
AIC.                     5323.7758000
BIC.                     5352.7168000
McFadden.R2.                0.2054148
Adj..McFadden.R2            0.2039195
Number.of.Observations.  2412.0000000
\end{CodeOutput}
\end{CodeChunk}

\textbf{Get the estimated model coefficients}:

\begin{CodeChunk}

\begin{CodeInput}
R> coef(mnl.wtp)
\end{CodeInput}

\begin{CodeOutput}
**Using results for model 10 of 10,
the best model (largest log-likelihood) from the multistart**
\end{CodeOutput}

\begin{CodeOutput}
    lambda       feat     dannon     hiland    yoplait 
 0.3665845  1.3405732  1.7490766 -8.3866503  3.7529018 
\end{CodeOutput}
\end{CodeChunk}

\textbf{Comparing WTP}:\\
Since WTP space models are non-convex, you cannot be certain that the
model reached a global solution, even when using a multistart. However,
homogeneous models in the preference space are convex, so you are
guaranteed to find the global solution in that space. Therefore, it can
be useful to compute the WTP from the preference space model and compare
it against the WTP from the WTP space model. If the WTP values and
log-likelhiood values from the two model spaces are equal, then the WTP
space model is likely at a global solution. To compare the WTP and
log-likelihood values between the preference space and WTP space models,
use the \texttt{wtpCompare()} function:

\begin{CodeChunk}

\begin{CodeInput}
R> wtpCompare(mnl.pref, mnl.wtp, priceName="price")
\end{CodeInput}

\begin{CodeOutput}
**Using results for model 10 of 10,
the best model (largest log-likelihood) from the multistart**
\end{CodeOutput}

\begin{CodeOutput}
                pref           wtp difference
lambda      0.366584     0.3665845   5.00e-07
feat        1.340569     1.3405732   4.21e-06
dannon      1.749081     1.7490766  -4.41e-06
hiland     -8.386651    -8.3866503   7.20e-07
yoplait     3.752907     3.7529018  -5.18e-06
logLik  -2656.887878 -2656.8878779   0.00e+00
\end{CodeOutput}
\end{CodeChunk}

\newpage

\hypertarget{mxl-model-in-the-preference-space}{%
\subsection{MXL model in the preference
space}\label{mxl-model-in-the-preference-space}}

Estimate the following mixed logit model in the preference space:

\input{./eqns/mxlPrefExample.Rmd}

where the parameters \(\alpha\), \(\beta_1\), \(\beta_2\), \(\beta_3\),
and \(\beta_4\) have units of utility, and the parameter for
\(x_{j}^{\mathrm{FEAT}}\) is normally distributed.

\textbf{Estimate the model}:

\begin{CodeChunk}

\begin{CodeInput}
R> library("logitr")
R> data(yogurt)
R> 
R> mxl.pref = logitr(
R>     data       = yogurt,
R>     choiceName = "choice",
R>     obsIDName  = "obsID",
R>     parNames   = c("price", "feat", "dannon", "hiland", "yoplait"),
R>     randPars   = c(feat="n"),
R>     options    = list(
R>     # You should run a multistart for MXL models since they are non-convex,
R>     # but it can take a long time. Here I just use 1 for brevity:
R>         numMultiStarts = 1,
R>         numDraws       = 500))
\end{CodeInput}
\end{CodeChunk}

\textbf{Print a summary of the results}:

\begin{CodeChunk}

\begin{CodeInput}
R> summary(mxl.pref)
\end{CodeInput}

\begin{CodeOutput}
=================================================
MODEL SUMMARY: 
                        
Model Space:  Preference
Model Run:        1 of 1
Iterations:           35
Elapsed Time:  0h:3m:11s

Model Coefficients: 
            Estimate StdError    tStat   pVal signif
price      -0.392769 0.026708 -14.7062 0.0000    ***
feat.mu     0.351935 0.204608   1.7201 0.0856      .
dannon      0.663864 0.055794  11.8985 0.0000    ***
hiland     -3.324274 0.164101 -20.2575 0.0000    ***
yoplait     1.458058 0.095513  15.2656 0.0000    ***
feat.sigma  2.360354 0.515567   4.5782 0.0000    ***
---
Signif. codes:  0 '***' 0.001 '**' 0.01 '*' 0.05 '.' 0.1 ' ' 1

Model Fit Values: 
                                     
Log.Likelihood.         -2645.9202707
Null.Log.Likelihood.    -3343.7419990
AIC.                     5303.8405000
BIC.                     5338.5698000
McFadden.R2.                0.2086948
Adj..McFadden.R2            0.2069005
Number.of.Observations.  2412.0000000

Summary of 10k Draws for Random Coefficients: 
       Min.   1st Qu.    Median      Mean  3rd Qu.     Max.
1 -8.822085 -1.240571 0.3511837 0.3499345 1.942272 8.875549
\end{CodeOutput}
\end{CodeChunk}

\textbf{Get the estimated model coefficients}:

\begin{CodeChunk}

\begin{CodeInput}
R> coef(mxl.pref)
\end{CodeInput}

\begin{CodeOutput}
     price    feat.mu     dannon     hiland    yoplait feat.sigma 
-0.3927685  0.3519352  0.6638642 -3.3242735  1.4580579  2.3603539 
\end{CodeOutput}
\end{CodeChunk}

\textbf{Get the WTP implied from the preference space model}:

\begin{CodeChunk}

\begin{CodeInput}
R> mxl.pref.wtp = wtp(mxl.pref, priceName="price")
R> mxl.pref.wtp
\end{CodeInput}

\begin{CodeOutput}
            Estimate StdError    tStat   pVal signif
lambda      0.392769 0.026786  14.6632 0.0000    ***
feat.mu     0.896037 0.539162   1.6619 0.0967      .
dannon      1.690217 0.195466   8.6471 0.0000    ***
hiland     -8.463696 0.525479 -16.1066 0.0000    ***
yoplait     3.712257 0.479741   7.7380 0.0000    ***
feat.sigma  6.009529 1.439662   4.1743 0.0000    ***
\end{CodeOutput}
\end{CodeChunk}

\newpage

\hypertarget{mxl-model-in-the-wtp-space}{%
\subsection{MXL model in the WTP
space}\label{mxl-model-in-the-wtp-space}}

Estimate the following mixed logit model in the WTP space:

\input{./eqns/mxlWtpExample.Rmd}

where the parameters \(\omega_1\), \(\omega_2\), \(\omega_3\), and
\(\omega_4\) have units of dollars and \(\lambda\) is the scale
parameter, and the WTP parameter for \(x_{j}^{\mathrm{FEAT}}\) is
normally distributed.

\textbf{Estimate the model}:

\begin{CodeChunk}

\begin{CodeInput}
R> library("logitr")
R> data(yogurt)
R> 
R> # Use WTP from preference space model as starting values for first run:
\end{CodeInput}
\end{CodeChunk}

\texttt{startingValues\ =\ mxl.pref.wtp\$Estimate}

\begin{CodeChunk}

\begin{CodeInput}
R> library("logitr")
R> data(yogurt)
R> 
R> mxl.wtp = logitr(
R>   data       = yogurt,
R>   choiceName = "choice",
R>   obsIDName  = "obsID",
R>   parNames   = c("feat", "dannon", "hiland", "yoplait"),
R>   priceName  = "price",
R>   randPars   = c(feat="n"),
R>   modelSpace = "wtp",
R>   options = list(
R>   # You should run a multistart for MXL models since they are non-convex,
R>   # but it can take a long time. Here I just use 1 for brevity:
R>     numMultiStarts = 1,
R>     startVals      = startingValues,
R>     startParBounds = c(-5,5),
R>     numDraws       = 500))
\end{CodeInput}
\end{CodeChunk}

\textbf{Print a summary of the results}:

\begin{CodeChunk}

\begin{CodeInput}
R> summary(mxl.wtp)
\end{CodeInput}

\begin{CodeOutput}
=================================================
MODEL SUMMARY: 
                                
Model Space:  Willingness-to-Pay
Model Run:                1 of 1
Iterations:                   14
Elapsed Time:          0h:1m:49s

Model Coefficients: 
            Estimate StdError    tStat  pVal signif
lambda      0.392827 0.026705  14.7101 0.000    ***
feat.mu     0.918960 0.535044   1.7175 0.086      .
dannon      1.690071 0.172643   9.7894 0.000    ***
hiland     -8.462586 0.517519 -16.3522 0.000    ***
yoplait     3.711836 0.161324  23.0086 0.000    ***
feat.sigma  6.024517 1.321772   4.5579 0.000    ***
---
Signif. codes:  0 '***' 0.001 '**' 0.01 '*' 0.05 '.' 0.1 ' ' 1

Model Fit Values: 
                                     
Log.Likelihood.         -2645.9049060
Null.Log.Likelihood.    -3343.7419990
AIC.                     5303.8098000
BIC.                     5338.5391000
McFadden.R2.                0.2086994
Adj..McFadden.R2            0.2069050
Number.of.Observations.  2412.0000000

Summary of 10k Draws for Random Coefficients: 
       Min.   1st Qu.    Median      Mean  3rd Qu.     Max.
1 -22.49661 -3.145719 0.9170417 0.9138531 4.978102 22.67445
\end{CodeOutput}
\end{CodeChunk}

\textbf{Get the estimated model coefficients}:

\begin{CodeChunk}

\begin{CodeInput}
R> coef(mxl.wtp)
\end{CodeInput}

\begin{CodeOutput}
    lambda    feat.mu     dannon     hiland    yoplait feat.sigma 
 0.3928271  0.9189597  1.6900713 -8.4625857  3.7118360  6.0245171 
\end{CodeOutput}
\end{CodeChunk}

\textbf{Comparing WTP}:\\
Note that the WTP will \textbf{not} necessarily be the same between
preference space and WTP space MXL models. This is because the
distributional assumptions in MXL models imply different distributions
on WTP depending on the model space. In this particular example, the
distributional assumptions are not too different and the WTP results are
similar. See Train and Weeks \citeyearpar{Train2005} and Sonnier,
Ainslie, and Otter \citeyearpar{Sonnier2007} for details on this topic:

\begin{CodeChunk}

\begin{CodeInput}
R> wtpCompare(mxl.pref, mxl.wtp, priceName="price")
\end{CodeInput}

\begin{CodeOutput}
                   pref           wtp  difference
lambda         0.392769     0.3928271  0.00005815
feat.mu        0.896037     0.9189597  0.02292273
dannon         1.690217     1.6900713 -0.00014566
hiland        -8.463696    -8.4625857  0.00111030
yoplait        3.712257     3.7118360 -0.00042097
feat.sigma     6.009529     6.0245171  0.01498813
logLik     -2645.920271 -2645.9049060  0.01536475
\end{CodeOutput}
\end{CodeChunk}

\newpage

\hypertarget{simulation-1}{%
\subsection{Simulation}\label{simulation-1}}

For a particular set of alternatives, simulate the expected shares given
an estimated model. First, create a set of altneratives (here I use one
particular choice set from the \texttt{Yogurt} data set):

\begin{CodeChunk}

\begin{CodeInput}
R> alts = subset(yogurt, obsID==42,
R+        select=c("feat", "price", "dannon", "hiland", "yoplait"))
R> row.names(alts) = c("dannon", "hiland", "weight", "yoplait")
R> alts
\end{CodeInput}

\begin{CodeOutput}
        feat price dannon hiland yoplait
dannon     0   6.3      1      0       0
hiland     1   6.1      0      1       0
weight     0   7.9      0      0       0
yoplait    0  11.5      0      0       1
\end{CodeOutput}
\end{CodeChunk}

\textbf{Run the simulation using the preference space MNL model}:

\begin{CodeChunk}

\begin{CodeInput}
R> mnl.pref.simulation = simulateShares(mnl.pref, alts)
R> mnl.pref.simulation
\end{CodeInput}

\begin{CodeOutput}
             share.mean  share.low share.high
Alt: dannon  0.60767184 0.54698563 0.65857677
Alt: hiland  0.02601784 0.01834165 0.03651461
Alt: weight  0.17802325 0.16322458 0.19246719
Alt: yoplait 0.18828707 0.13935579 0.25079772
\end{CodeOutput}
\end{CodeChunk}

\textbf{Run the simulation using the WTP space MNL model} (note that you
must denote the ``price'' variable):

\begin{CodeChunk}

\begin{CodeInput}
R> mnl.wtp.simulation = simulateShares(mnl.wtp, alts, priceName="price")
\end{CodeInput}

\begin{CodeOutput}
**Using results for model 10 of 10,
the best model (largest log-likelihood) from the multistart**
\end{CodeOutput}

\begin{CodeInput}
R> mnl.wtp.simulation
\end{CodeInput}

\begin{CodeOutput}
             share.mean  share.low share.high
Alt: dannon   0.6076718 0.55511076 0.65976263
Alt: hiland   0.0260179 0.01796662 0.03759781
Alt: weight   0.1780234 0.14754337 0.20826982
Alt: yoplait  0.1882869 0.16402123 0.21154407
\end{CodeOutput}
\end{CodeChunk}

\textbf{Run the simulation using the preference space MXL model}:

\begin{CodeChunk}

\begin{CodeInput}
R> mxl.pref.simulation = simulateShares(mxl.pref, alts)
\end{CodeInput}
\end{CodeChunk}
\begin{CodeChunk}

\begin{CodeInput}
R> mxl.pref.simulation
\end{CodeInput}

\begin{CodeOutput}
             share.mean  share.low share.high
Alt: dannon  0.58484934 0.50533529  0.6542321
Alt: hiland  0.08666403 0.03025259  0.1651975
Alt: weight  0.16062305 0.14149715  0.1781776
Alt: yoplait 0.16786359 0.11902155  0.2273408
\end{CodeOutput}
\end{CodeChunk}

\textbf{Run the simulation using the WTP space MXL model} (note that you
must denote the ``price'' variable):

\begin{CodeChunk}

\begin{CodeInput}
R> mxl.wtp.simulation = simulateShares(mxl.wtp, alts, priceName="price")
\end{CodeInput}
\end{CodeChunk}
\begin{CodeChunk}

\begin{CodeInput}
R> mxl.wtp.simulation
\end{CodeInput}

\begin{CodeOutput}
             share.mean  share.low share.high
Alt: dannon  0.58435544 0.52665244  0.6387125
Alt: hiland  0.08750646 0.03109963  0.1671529
Alt: weight  0.16046566 0.12583610  0.1964370
Alt: yoplait 0.16767244 0.13773861  0.1964418
\end{CodeOutput}
\end{CodeChunk}

\bibliography{library.bib}


\end{document}

